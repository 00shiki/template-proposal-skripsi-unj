\usepackage{geometry}
\geometry{verbose,tmargin=3cm,bmargin=3cm,lmargin=4cm,rmargin=3cm}
\usepackage[pdftex,bookmarks=true,unicode=true,pdfusetitle,bookmarksnumbered=true,bookmarksopen=true,breaklinks=true,pdfborder={0 0 1},backref=true,colorlinks=false]{hyperref}

%%%% identitas (kalo file pdf nya di klik kanan, info..)
\hypersetup{pdfauthor={\penulis}, 
	pdfsubject={SKRIPSI},
	pdftitle={\judul},
	pdfkeywords={\keywords}}

\usepackage[htt]{hyphenat}
\usepackage{listing}
\usepackage{graphicx}      %menghandle gambar
\usepackage{tabularx}
\usepackage{float}
\usepackage[hang,nooneline,scriptsize,md]{subfigure}
\usepackage{epsfig}
\usepackage[font=small,labelfont=bf]{caption}
\usepackage[onehalfspacing]{setspace}      %buat kombinasi spacing
\usepackage{indentfirst}
%\setlength{\parindent}{0.6cm}
\usepackage{parskip}
\usepackage[titletoc]{appendix}
\usepackage{cite}          % buat cite pake BibTex
\bibliographystyle{apalike}
\usepackage[bahasa]{babel} %plugin bahasa 

\usepackage[utf8]{inputenc}
\usepackage[T1]{fontenc}   % biar bisa kombinasi font
\usepackage{mathptmx}
\usepackage{amssymb}
\usepackage{amsmath}
\usepackage{mathrsfs}
\usepackage{dsfont}
\usepackage{listings}
\usepackage{relsize}
%%% pengaturan global
\graphicspath{{gambar/}}   % letak direktori penyimpanan gambar
%\usepackage[authoryear,round]{natbib}

%\citestyle{nature}


%\usepackage[fixlanguage]{babelbib}
%\selectbiblanguage{bahasa}

\pagestyle{plain}
\setlength{\parindent}{20pt}

%format judul
\usepackage{titlesec}
\titleformat{\chapter}[display]
{\bfseries\Large}
{%
	% \vskip-5em
	\filcenter
	\LARGE\MakeUppercase \chaptertitlename\ 
	\LARGE\thechapter}
{0mm}
{\filcenter}
\titleformat*{\section}{\bfseries\large}
\titleformat*{\subsection}{\bfseries}
\titlespacing*{\chapter}{0pt}{-1.5\baselineskip}{1em}

%menambahkan titik titik di daftar isi, gambar, tabel
\usepackage[subfigure]{tocloft}
\renewcommand{\cftpartleader}{\dotfill}
\renewcommand{\cftpartafterpnum}{\cftparfillskip}
\renewcommand{\cftchapleader}{\dotfill} 
\renewcommand{\cftsecleader}{\dotfill}
\renewcommand{\cftsubsecleader}{\dotfill}

%tambah "bab" x di daftar isi 
\renewcommand*\cftchappresnum{\MakeUppercase{bab}~}
\renewcommand\chaptername{BAB}
\settowidth{\cftchapnumwidth}{\cftchappresnum}
\renewcommand{\cftchapaftersnumb}{\quad}
\addtocontents{toc}{
	%\renewcommand\protect*\protect\cftchappresnum{\MakeUppercase{\chaptername}~}}
	\protect\renewcommand*\protect\cftchappresnum{\MakeUppercase{\chaptername}~}}

%Penomoran Bab berupa bilangan romawi
\renewcommand{\thechapter}{\Roman{chapter}}%

%tambah "gambar" dan "table" di daftar gambar dan daftar tabel
\usepackage{chngcntr}  
%\counterwithout{figure}{chapter}
\renewcommand{\cftfigpresnum}{\bfseries Gambar\ }
\renewcommand{\cfttabpresnum}{\bfseries Tabel\ }
\newlength{\mylenf}
\settowidth{\mylenf}{\cftfigpresnum}
\setlength{\cftfignumwidth}{\dimexpr\mylenf+2em}
\settowidth{\mylenf}{\cfttabpresnum}
\setlength{\cfttabnumwidth}{\dimexpr\mylenf+2em}
\makeatletter

%bikin listoftable tanpa judul, gara2 aneh
%\counterwithout{table}{chapter}
\newcommand\daftartabel{%
	\phantomsection
	\@starttoc{lof}%
	\bigskip
	\@starttoc{lot}}
\makeatother



%ilangin judul bibli 
\makeatletter
\renewenvironment{thebibliography}[1]{%
	\section*{\refname}%
	\@mkboth{\MakeUppercase\refname}{\MakeUppercase\refname}%
	\list{\@biblabel{\@arabic\c@enumiv}}%
	{\settowidth\labelwidth{\@biblabel{#1}}%
		\leftmargin\labelwidth
		\advance\leftmargin\labelsep
		\@openbib@code
		\usecounter{enumiv}%
		\let\p@enumiv\@empty
		\renewcommand\theenumiv{\@arabic\c@enumiv}}%
	\sloppy
	\clubpenalty4000
	\@clubpenalty \clubpenalty
	\widowpenalty4000%
	\sfcode`\.\@m}
{\def\@noitemerr
	{\@latex@warning{Empty `thebibliography' environment}}%
	\endlist}
\makeatother


%lampiran
\makeatletter
\newcommand\appendix@chapter[1]{%
	\refstepcounter{chapter}%
	\orig@chapter*{LAMPIRAN \@Alph\c@chapter \\  #1}\vspace{1.5em}%
	\addcontentsline{toc}{chapter}{LAMPIRAN \@Alph\c@chapter: #1}%
}
\let\orig@chapter\chapter
\g@addto@macro\appendix{\let\chapter\appendix@chapter}
\makeatother